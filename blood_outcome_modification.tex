% Options for packages loaded elsewhere
\PassOptionsToPackage{unicode}{hyperref}
\PassOptionsToPackage{hyphens}{url}
%
\documentclass[
]{article}
\usepackage{amsmath,amssymb}
\usepackage{lmodern}
\usepackage{iftex}
\ifPDFTeX
  \usepackage[T1]{fontenc}
  \usepackage[utf8]{inputenc}
  \usepackage{textcomp} % provide euro and other symbols
\else % if luatex or xetex
  \usepackage{unicode-math}
  \defaultfontfeatures{Scale=MatchLowercase}
  \defaultfontfeatures[\rmfamily]{Ligatures=TeX,Scale=1}
\fi
% Use upquote if available, for straight quotes in verbatim environments
\IfFileExists{upquote.sty}{\usepackage{upquote}}{}
\IfFileExists{microtype.sty}{% use microtype if available
  \usepackage[]{microtype}
  \UseMicrotypeSet[protrusion]{basicmath} % disable protrusion for tt fonts
}{}
\makeatletter
\@ifundefined{KOMAClassName}{% if non-KOMA class
  \IfFileExists{parskip.sty}{%
    \usepackage{parskip}
  }{% else
    \setlength{\parindent}{0pt}
    \setlength{\parskip}{6pt plus 2pt minus 1pt}}
}{% if KOMA class
  \KOMAoptions{parskip=half}}
\makeatother
\usepackage{xcolor}
\usepackage[margin=1in]{geometry}
\usepackage{color}
\usepackage{fancyvrb}
\newcommand{\VerbBar}{|}
\newcommand{\VERB}{\Verb[commandchars=\\\{\}]}
\DefineVerbatimEnvironment{Highlighting}{Verbatim}{commandchars=\\\{\}}
% Add ',fontsize=\small' for more characters per line
\usepackage{framed}
\definecolor{shadecolor}{RGB}{248,248,248}
\newenvironment{Shaded}{\begin{snugshade}}{\end{snugshade}}
\newcommand{\AlertTok}[1]{\textcolor[rgb]{0.94,0.16,0.16}{#1}}
\newcommand{\AnnotationTok}[1]{\textcolor[rgb]{0.56,0.35,0.01}{\textbf{\textit{#1}}}}
\newcommand{\AttributeTok}[1]{\textcolor[rgb]{0.77,0.63,0.00}{#1}}
\newcommand{\BaseNTok}[1]{\textcolor[rgb]{0.00,0.00,0.81}{#1}}
\newcommand{\BuiltInTok}[1]{#1}
\newcommand{\CharTok}[1]{\textcolor[rgb]{0.31,0.60,0.02}{#1}}
\newcommand{\CommentTok}[1]{\textcolor[rgb]{0.56,0.35,0.01}{\textit{#1}}}
\newcommand{\CommentVarTok}[1]{\textcolor[rgb]{0.56,0.35,0.01}{\textbf{\textit{#1}}}}
\newcommand{\ConstantTok}[1]{\textcolor[rgb]{0.00,0.00,0.00}{#1}}
\newcommand{\ControlFlowTok}[1]{\textcolor[rgb]{0.13,0.29,0.53}{\textbf{#1}}}
\newcommand{\DataTypeTok}[1]{\textcolor[rgb]{0.13,0.29,0.53}{#1}}
\newcommand{\DecValTok}[1]{\textcolor[rgb]{0.00,0.00,0.81}{#1}}
\newcommand{\DocumentationTok}[1]{\textcolor[rgb]{0.56,0.35,0.01}{\textbf{\textit{#1}}}}
\newcommand{\ErrorTok}[1]{\textcolor[rgb]{0.64,0.00,0.00}{\textbf{#1}}}
\newcommand{\ExtensionTok}[1]{#1}
\newcommand{\FloatTok}[1]{\textcolor[rgb]{0.00,0.00,0.81}{#1}}
\newcommand{\FunctionTok}[1]{\textcolor[rgb]{0.00,0.00,0.00}{#1}}
\newcommand{\ImportTok}[1]{#1}
\newcommand{\InformationTok}[1]{\textcolor[rgb]{0.56,0.35,0.01}{\textbf{\textit{#1}}}}
\newcommand{\KeywordTok}[1]{\textcolor[rgb]{0.13,0.29,0.53}{\textbf{#1}}}
\newcommand{\NormalTok}[1]{#1}
\newcommand{\OperatorTok}[1]{\textcolor[rgb]{0.81,0.36,0.00}{\textbf{#1}}}
\newcommand{\OtherTok}[1]{\textcolor[rgb]{0.56,0.35,0.01}{#1}}
\newcommand{\PreprocessorTok}[1]{\textcolor[rgb]{0.56,0.35,0.01}{\textit{#1}}}
\newcommand{\RegionMarkerTok}[1]{#1}
\newcommand{\SpecialCharTok}[1]{\textcolor[rgb]{0.00,0.00,0.00}{#1}}
\newcommand{\SpecialStringTok}[1]{\textcolor[rgb]{0.31,0.60,0.02}{#1}}
\newcommand{\StringTok}[1]{\textcolor[rgb]{0.31,0.60,0.02}{#1}}
\newcommand{\VariableTok}[1]{\textcolor[rgb]{0.00,0.00,0.00}{#1}}
\newcommand{\VerbatimStringTok}[1]{\textcolor[rgb]{0.31,0.60,0.02}{#1}}
\newcommand{\WarningTok}[1]{\textcolor[rgb]{0.56,0.35,0.01}{\textbf{\textit{#1}}}}
\usepackage{graphicx}
\makeatletter
\def\maxwidth{\ifdim\Gin@nat@width>\linewidth\linewidth\else\Gin@nat@width\fi}
\def\maxheight{\ifdim\Gin@nat@height>\textheight\textheight\else\Gin@nat@height\fi}
\makeatother
% Scale images if necessary, so that they will not overflow the page
% margins by default, and it is still possible to overwrite the defaults
% using explicit options in \includegraphics[width, height, ...]{}
\setkeys{Gin}{width=\maxwidth,height=\maxheight,keepaspectratio}
% Set default figure placement to htbp
\makeatletter
\def\fps@figure{htbp}
\makeatother
\setlength{\emergencystretch}{3em} % prevent overfull lines
\providecommand{\tightlist}{%
  \setlength{\itemsep}{0pt}\setlength{\parskip}{0pt}}
\setcounter{secnumdepth}{-\maxdimen} % remove section numbering
\ifLuaTeX
  \usepackage{selnolig}  % disable illegal ligatures
\fi
\IfFileExists{bookmark.sty}{\usepackage{bookmark}}{\usepackage{hyperref}}
\IfFileExists{xurl.sty}{\usepackage{xurl}}{} % add URL line breaks if available
\urlstyle{same} % disable monospaced font for URLs
\hypersetup{
  pdftitle={blood outcomes modification},
  hidelinks,
  pdfcreator={LaTeX via pandoc}}

\title{blood outcomes modification}
\author{}
\date{\vspace{-2.5em}}

\begin{document}
\maketitle

\hypertarget{load-useful-packages}{%
\subsection{load useful packages}\label{load-useful-packages}}

\begin{Shaded}
\begin{Highlighting}[]
\CommentTok{\# load the useful packages}
\FunctionTok{library}\NormalTok{(dplyr)}
\FunctionTok{library}\NormalTok{(plotly)}
\FunctionTok{library}\NormalTok{(CBCgrps)}
\FunctionTok{library}\NormalTok{(networkD3)}
\end{Highlighting}
\end{Shaded}

\hypertarget{import-the-data-and-then-have-a-look}{%
\subsection{import the data and then have a
look}\label{import-the-data-and-then-have-a-look}}

\begin{Shaded}
\begin{Highlighting}[]
\NormalTok{raw\_data }\OtherTok{=} \FunctionTok{read.csv}\NormalTok{(}\StringTok{"outputs/raw\_data.csv"}\NormalTok{, }\AttributeTok{encoding=}\StringTok{"UTF{-}8"}\NormalTok{)}
\NormalTok{raw\_data[raw\_data}\SpecialCharTok{==}\StringTok{""}\NormalTok{] }\OtherTok{=} \ConstantTok{NA} \CommentTok{\# convert the null values to NA}
\end{Highlighting}
\end{Shaded}

have a look of the structure of dataframe

\begin{Shaded}
\begin{Highlighting}[]
\NormalTok{raw\_data }\OtherTok{=}\NormalTok{ raw\_data[,}\DecValTok{1}\SpecialCharTok{:}\DecValTok{78}\NormalTok{]}
\NormalTok{raw\_data }\SpecialCharTok{\%\textgreater{}\%} \FunctionTok{dim}\NormalTok{()}
\end{Highlighting}
\end{Shaded}

\begin{verbatim}
## [1] 116  78
\end{verbatim}

\hypertarget{ux5904ux7406ux51e0ux4e2aux5173ux952eux7684ux53d8ux91cf}{%
\subsection{处理几个关键的变量}\label{ux5904ux7406ux51e0ux4e2aux5173ux952eux7684ux53d8ux91cf}}

\hypertarget{ux9996ux5148ux662fux8d2bux8840ux7ed3ux5c40ux548cux7a0bux5ea6ux8981ux8003ux8651ux628aux5b83ux7f16ux7801.}{%
\subsubsection{首先是贫血结局和程度,要考虑把它编码.}\label{ux9996ux5148ux662fux8d2bux8840ux7ed3ux5c40ux548cux7a0bux5ea6ux8981ux8003ux8651ux628aux5b83ux7f16ux7801.}}

将贫血与否先编码为0或者1

\begin{Shaded}
\begin{Highlighting}[]
\NormalTok{blood\_outcome }\OtherTok{=}\NormalTok{ raw\_data[,}\FunctionTok{c}\NormalTok{(}\StringTok{"姓名"}\NormalTok{,}\StringTok{"是否贫血"}\NormalTok{,}\StringTok{"是否贫血.1"}\NormalTok{)]}

\NormalTok{blood\_outcome}\SpecialCharTok{$}\NormalTok{是否贫血 }\OtherTok{=} \FunctionTok{ifelse}\NormalTok{(raw\_data}\SpecialCharTok{$}\NormalTok{是否贫血}\SpecialCharTok{==}\StringTok{"有"}\NormalTok{, }\DecValTok{1}\NormalTok{, }\DecValTok{0}\NormalTok{)   }
\NormalTok{blood\_outcome}\SpecialCharTok{$}\NormalTok{是否贫血}\FloatTok{.1} \OtherTok{=} \FunctionTok{ifelse}\NormalTok{(raw\_data}\SpecialCharTok{$}\NormalTok{是否贫血}\FloatTok{.1}\SpecialCharTok{==}\StringTok{"有"}\NormalTok{, }\DecValTok{1}\NormalTok{, }\DecValTok{0}\NormalTok{)   }
\end{Highlighting}
\end{Shaded}

看看前五行

\begin{Shaded}
\begin{Highlighting}[]
\NormalTok{blood\_outcome }\SpecialCharTok{\%\textgreater{}\%} \FunctionTok{head}\NormalTok{()}
\end{Highlighting}
\end{Shaded}

\begin{verbatim}
##     姓名 是否贫血 是否贫血.1
## 1   谷巍        1          1
## 2 武艳丽        0          0
## 3   刘静        0          0
## 4 王依菡        1          0
## 5 李凤章        1          0
## 6 王习俭        1          1
\end{verbatim}

将变化搞出来,即两者相减,是否改善,如果是小于等于0,则为0,如果是大于0,则为1,即改善了。

\begin{Shaded}
\begin{Highlighting}[]
\NormalTok{blood\_outcome[}\StringTok{"是否贫血变化"}\NormalTok{] }\OtherTok{=}\NormalTok{ blood\_outcome}\SpecialCharTok{$}\NormalTok{是否贫血 }\SpecialCharTok{{-}}\NormalTok{ blood\_outcome}\SpecialCharTok{$}\NormalTok{是否贫血}\FloatTok{.1}
\NormalTok{blood\_outcome}\SpecialCharTok{$}\NormalTok{是否贫血变化 }\OtherTok{=} \FunctionTok{ifelse}\NormalTok{(blood\_outcome}\SpecialCharTok{$}\NormalTok{是否贫血变化}\SpecialCharTok{\textgreater{}}\DecValTok{0}\NormalTok{, }\DecValTok{1}\NormalTok{, }\DecValTok{0}\NormalTok{)}
\NormalTok{blood\_outcome }\SpecialCharTok{\%\textgreater{}\%} \FunctionTok{head}\NormalTok{(}\DecValTok{10}\NormalTok{)}
\end{Highlighting}
\end{Shaded}

\begin{verbatim}
##      姓名 是否贫血 是否贫血.1 是否贫血变化
## 1    谷巍        1          1            0
## 2  武艳丽        0          0            0
## 3    刘静        0          0            0
## 4  王依菡        1          0            1
## 5  李凤章        1          0            1
## 6  王习俭        1          1            0
## 7  朱德顺        1          0            1
## 8  卢兴福        0          0            0
## 9    李泽        1          0            1
## 10 刘淑琴        1          0            1
\end{verbatim}

初步看到,如果使用两分类的变量,可以看到不少贫血情况随着时间恶化的人,这不好。

那么,有多少人改善呢?

\begin{Shaded}
\begin{Highlighting}[]
\NormalTok{blood\_outcome}\SpecialCharTok{$}\NormalTok{是否贫血变化 }\SpecialCharTok{\%\textgreater{}\%} \FunctionTok{table}\NormalTok{()}
\end{Highlighting}
\end{Shaded}

\begin{verbatim}
## .
##  0  1 
## 82 34
\end{verbatim}

看到贫血到不贫血的有34人,占比29\%

\hypertarget{ux9700ux8981ux8003ux5bdfux8d2bux8840ux7b49ux7ea7}{%
\subsubsection{需要考察贫血等级}\label{ux9700ux8981ux8003ux5bdfux8d2bux8840ux7b49ux7ea7}}

先将等级编码

\begin{Shaded}
\begin{Highlighting}[]
\NormalTok{blood\_level }\OtherTok{=}\NormalTok{ raw\_data[,}\FunctionTok{c}\NormalTok{(}\StringTok{"姓名"}\NormalTok{,}\StringTok{"贫血程度"}\NormalTok{,}\StringTok{"贫血程度.1"}\NormalTok{)]}
\NormalTok{blood\_level }\SpecialCharTok{\%\textgreater{}\%} \FunctionTok{head}\NormalTok{()}
\end{Highlighting}
\end{Shaded}

\begin{verbatim}
##     姓名 贫血程度 贫血程度.1
## 1   谷巍 轻度贫血       轻度
## 2 武艳丽     正常       正常
## 3   刘静     正常       正常
## 4 王依菡 轻度贫血       正常
## 5 李凤章 轻度贫血       正常
## 6 王习俭 中度贫血       轻度
\end{verbatim}

先编码基线的贫血程度

\begin{Shaded}
\begin{Highlighting}[]
\NormalTok{blood\_level}\SpecialCharTok{$}\NormalTok{贫血程度 }\OtherTok{=} \FunctionTok{replace}\NormalTok{(blood\_level}\SpecialCharTok{$}\NormalTok{贫血程度, raw\_data}\SpecialCharTok{$}\NormalTok{贫血程度}\SpecialCharTok{==}\StringTok{"正常"}\NormalTok{, }\DecValTok{0}\NormalTok{)}
\NormalTok{blood\_level}\SpecialCharTok{$}\NormalTok{贫血程度 }\OtherTok{=} \FunctionTok{replace}\NormalTok{(blood\_level}\SpecialCharTok{$}\NormalTok{贫血程度, raw\_data}\SpecialCharTok{$}\NormalTok{贫血程度}\SpecialCharTok{==}\StringTok{"轻度贫血"}\NormalTok{, }\DecValTok{1}\NormalTok{)}
\NormalTok{blood\_level}\SpecialCharTok{$}\NormalTok{贫血程度 }\OtherTok{=} \FunctionTok{replace}\NormalTok{(blood\_level}\SpecialCharTok{$}\NormalTok{贫血程度, raw\_data}\SpecialCharTok{$}\NormalTok{贫血程度}\SpecialCharTok{==}\StringTok{"中度贫血"}\NormalTok{, }\DecValTok{2}\NormalTok{)}
\NormalTok{blood\_level}\SpecialCharTok{$}\NormalTok{贫血程度 }\OtherTok{=} \FunctionTok{replace}\NormalTok{(blood\_level}\SpecialCharTok{$}\NormalTok{贫血程度, raw\_data}\SpecialCharTok{$}\NormalTok{贫血程度}\SpecialCharTok{==}\StringTok{"重度贫血"}\NormalTok{, }\DecValTok{3}\NormalTok{)}
\NormalTok{blood\_level}\SpecialCharTok{$}\NormalTok{贫血程度 }\OtherTok{=} \FunctionTok{replace}\NormalTok{(blood\_level}\SpecialCharTok{$}\NormalTok{贫血程度, raw\_data}\SpecialCharTok{$}\NormalTok{贫血程度}\SpecialCharTok{==}\StringTok{"极重度贫血"}\NormalTok{, }\DecValTok{4}\NormalTok{)}
\NormalTok{blood\_level }\SpecialCharTok{\%\textgreater{}\%} \FunctionTok{head}\NormalTok{(}\DecValTok{10}\NormalTok{)}
\end{Highlighting}
\end{Shaded}

\begin{verbatim}
##      姓名 贫血程度 贫血程度.1
## 1    谷巍        1       轻度
## 2  武艳丽        0       正常
## 3    刘静        0       正常
## 4  王依菡        1       正常
## 5  李凤章        1       正常
## 6  王习俭        2       轻度
## 7  朱德顺        1       正常
## 8  卢兴福        0       正常
## 9    李泽        1       正常
## 10 刘淑琴        2       正常
\end{verbatim}

然后编码随访的贫血程度

\begin{Shaded}
\begin{Highlighting}[]
\NormalTok{blood\_level}\SpecialCharTok{$}\NormalTok{贫血程度}\FloatTok{.1} \SpecialCharTok{\%\textgreater{}\%} \FunctionTok{unique}\NormalTok{()}
\end{Highlighting}
\end{Shaded}

\begin{verbatim}
## [1] "轻度"   "正常"   "中度"   "重度"   "极重度"
\end{verbatim}

\begin{Shaded}
\begin{Highlighting}[]
\NormalTok{blood\_level}\SpecialCharTok{$}\NormalTok{贫血程度}\FloatTok{.1} \OtherTok{=} \FunctionTok{replace}\NormalTok{(blood\_level}\SpecialCharTok{$}\NormalTok{贫血程度}\FloatTok{.1}\NormalTok{, raw\_data}\SpecialCharTok{$}\NormalTok{贫血程度}\FloatTok{.1}\SpecialCharTok{==}\StringTok{"正常"}\NormalTok{, }\DecValTok{0}\NormalTok{)}
\NormalTok{blood\_level}\SpecialCharTok{$}\NormalTok{贫血程度}\FloatTok{.1} \OtherTok{=} \FunctionTok{replace}\NormalTok{(blood\_level}\SpecialCharTok{$}\NormalTok{贫血程度}\FloatTok{.1}\NormalTok{, raw\_data}\SpecialCharTok{$}\NormalTok{贫血程度}\FloatTok{.1}\SpecialCharTok{==}\StringTok{"轻度"}\NormalTok{, }\DecValTok{1}\NormalTok{)}
\NormalTok{blood\_level}\SpecialCharTok{$}\NormalTok{贫血程度}\FloatTok{.1} \OtherTok{=} \FunctionTok{replace}\NormalTok{(blood\_level}\SpecialCharTok{$}\NormalTok{贫血程度}\FloatTok{.1}\NormalTok{, raw\_data}\SpecialCharTok{$}\NormalTok{贫血程度}\FloatTok{.1}\SpecialCharTok{==}\StringTok{"中度"}\NormalTok{, }\DecValTok{2}\NormalTok{)}
\NormalTok{blood\_level}\SpecialCharTok{$}\NormalTok{贫血程度}\FloatTok{.1} \OtherTok{=} \FunctionTok{replace}\NormalTok{(blood\_level}\SpecialCharTok{$}\NormalTok{贫血程度}\FloatTok{.1}\NormalTok{, raw\_data}\SpecialCharTok{$}\NormalTok{贫血程度}\FloatTok{.1}\SpecialCharTok{==}\StringTok{"重度"}\NormalTok{, }\DecValTok{3}\NormalTok{)}
\NormalTok{blood\_level}\SpecialCharTok{$}\NormalTok{贫血程度}\FloatTok{.1} \OtherTok{=} \FunctionTok{replace}\NormalTok{(blood\_level}\SpecialCharTok{$}\NormalTok{贫血程度}\FloatTok{.1}\NormalTok{, raw\_data}\SpecialCharTok{$}\NormalTok{贫血程度}\FloatTok{.1}\SpecialCharTok{==}\StringTok{"极重度"}\NormalTok{, }\DecValTok{4}\NormalTok{)}
\NormalTok{blood\_level }\SpecialCharTok{\%\textgreater{}\%} \FunctionTok{head}\NormalTok{(}\DecValTok{10}\NormalTok{)}
\end{Highlighting}
\end{Shaded}

\begin{verbatim}
##      姓名 贫血程度 贫血程度.1
## 1    谷巍        1          1
## 2  武艳丽        0          0
## 3    刘静        0          0
## 4  王依菡        1          0
## 5  李凤章        1          0
## 6  王习俭        2          1
## 7  朱德顺        1          0
## 8  卢兴福        0          0
## 9    李泽        1          0
## 10 刘淑琴        2          0
\end{verbatim}

将变化搞出来,即两者相减,是否改善,如果是小于等于0,则为0,如果是大于0,则为1,即改善了。

\begin{Shaded}
\begin{Highlighting}[]
\NormalTok{blood\_level}\SpecialCharTok{$}\NormalTok{贫血改善幅度 }\OtherTok{=} \FunctionTok{as.integer}\NormalTok{(blood\_level}\SpecialCharTok{$}\NormalTok{贫血程度) }\SpecialCharTok{{-}} \FunctionTok{as.integer}\NormalTok{(blood\_level}\SpecialCharTok{$}\NormalTok{贫血程度}\FloatTok{.1}\NormalTok{)}
\end{Highlighting}
\end{Shaded}

统计下贫血改善有多少人

\begin{Shaded}
\begin{Highlighting}[]
\NormalTok{blood\_level}\SpecialCharTok{$}\NormalTok{贫血改善幅度 }\SpecialCharTok{\%\textgreater{}\%} \FunctionTok{table}\NormalTok{()}
\end{Highlighting}
\end{Shaded}

\begin{verbatim}
## .
## -2 -1  0  1  2  3 
##  1 13 39 45 14  4
\end{verbatim}

将改善幅度改成0和1,0对应小于等于0的值,1对应大于等于1的值,叫做``贫血幅度是否改善''。

\begin{Shaded}
\begin{Highlighting}[]
\CommentTok{\# 贫血幅度是否改善}
\NormalTok{blood\_level}\SpecialCharTok{$}\NormalTok{贫血幅度是否改善 }\OtherTok{=} \FunctionTok{ifelse}\NormalTok{(blood\_level}\SpecialCharTok{$}\NormalTok{贫血改善幅度}\SpecialCharTok{\textgreater{}}\DecValTok{0}\NormalTok{, }\DecValTok{1}\NormalTok{, }\DecValTok{0}\NormalTok{)}
\NormalTok{blood\_level }\SpecialCharTok{\%\textgreater{}\%} \FunctionTok{head}\NormalTok{(}\DecValTok{10}\NormalTok{)}
\end{Highlighting}
\end{Shaded}

\begin{verbatim}
##      姓名 贫血程度 贫血程度.1 贫血改善幅度 贫血幅度是否改善
## 1    谷巍        1          1            0                0
## 2  武艳丽        0          0            0                0
## 3    刘静        0          0            0                0
## 4  王依菡        1          0            1                1
## 5  李凤章        1          0            1                1
## 6  王习俭        2          1            1                1
## 7  朱德顺        1          0            1                1
## 8  卢兴福        0          0            0                0
## 9    李泽        1          0            1                1
## 10 刘淑琴        2          0            2                1
\end{verbatim}

\begin{Shaded}
\begin{Highlighting}[]
\NormalTok{blood\_level}\SpecialCharTok{$}\NormalTok{贫血幅度是否改善 }\SpecialCharTok{\%\textgreater{}\%} \FunctionTok{table}\NormalTok{()}
\end{Highlighting}
\end{Shaded}

\begin{verbatim}
## .
##  0  1 
## 53 63
\end{verbatim}

\hypertarget{ux9700ux8981ux8003ux5bdfux8840ux7ea2ux86cbux767dux60c5ux51b5}{%
\subsubsection{需要考察血红蛋白情况}\label{ux9700ux8981ux8003ux5bdfux8840ux7ea2ux86cbux767dux60c5ux51b5}}

血红蛋白是数值型数据,不需要额外编码。

\begin{Shaded}
\begin{Highlighting}[]
\NormalTok{raw\_data}\SpecialCharTok{$}\NormalTok{血红蛋白 }\SpecialCharTok{\%\textgreater{}\%} \FunctionTok{head}\NormalTok{(}\DecValTok{10}\NormalTok{)}
\end{Highlighting}
\end{Shaded}

\begin{verbatim}
##  [1] 100 155 120 106 100  87 103 119 109  88
\end{verbatim}

将血红蛋白的数据拉出来,检查缺失值

\begin{Shaded}
\begin{Highlighting}[]
\NormalTok{hemoglobin }\OtherTok{=}\NormalTok{ raw\_data[, }\FunctionTok{c}\NormalTok{(}\StringTok{"姓名"}\NormalTok{, }\StringTok{"血红蛋白"}\NormalTok{, }\StringTok{"血红蛋白.1"}\NormalTok{)]}
\NormalTok{hemoglobin }\SpecialCharTok{\%\textgreater{}\%} \FunctionTok{head}\NormalTok{(}\DecValTok{10}\NormalTok{)}
\end{Highlighting}
\end{Shaded}

\begin{verbatim}
##      姓名 血红蛋白 血红蛋白.1
## 1    谷巍      100        116
## 2  武艳丽      155        124
## 3    刘静      120        121
## 4  王依菡      106        121
## 5  李凤章      100        116
## 6  王习俭       87        108
## 7  朱德顺      103        119
## 8  卢兴福      119        118
## 9    李泽      109        116
## 10 刘淑琴       88        120
\end{verbatim}

\begin{Shaded}
\begin{Highlighting}[]
\NormalTok{hemoglobin}\SpecialCharTok{$}\NormalTok{血红蛋白数值变化 }\OtherTok{=}\NormalTok{ hemoglobin}\SpecialCharTok{$}\NormalTok{血红蛋白}\FloatTok{.1} \SpecialCharTok{{-}}\NormalTok{hemoglobin}\SpecialCharTok{$}\NormalTok{血红蛋白}
\NormalTok{hemoglobin}\SpecialCharTok{$}\NormalTok{血红蛋白是否改善 }\OtherTok{=} \FunctionTok{ifelse}\NormalTok{(hemoglobin}\SpecialCharTok{$}\NormalTok{血红蛋白}\FloatTok{.1} \SpecialCharTok{\textgreater{}}\NormalTok{ hemoglobin}\SpecialCharTok{$}\NormalTok{血红蛋白, }\DecValTok{1}\NormalTok{, }\DecValTok{0}\NormalTok{)}

\NormalTok{hemoglobin }\SpecialCharTok{\%\textgreater{}\%} \FunctionTok{head}\NormalTok{(}\DecValTok{20}\NormalTok{)}
\end{Highlighting}
\end{Shaded}

\begin{verbatim}
##      姓名 血红蛋白 血红蛋白.1 血红蛋白数值变化 血红蛋白是否改善
## 1    谷巍      100        116               16                1
## 2  武艳丽      155        124              -31                0
## 3    刘静      120        121                1                1
## 4  王依菡      106        121               15                1
## 5  李凤章      100        116               16                1
## 6  王习俭       87        108               21                1
## 7  朱德顺      103        119               16                1
## 8  卢兴福      119        118               -1                0
## 9    李泽      109        116                7                1
## 10 刘淑琴       88        120               32                1
## 11 范木英      100        114               14                1
## 12 刘清儒      116        119                3                1
## 13 韩国珍      102        155               53                1
## 14   陈南      106        120               14                1
## 15 杨连华      110        119                9                1
## 16 王惠玲      129        108              -21                0
## 17 姜力华       75        120               45                1
## 18 李惠鸾       96        119               23                1
## 19 李清文       50         91               41                1
## 20 高树芹       43        107               64                1
\end{verbatim}

考察数值的分布,目前来说这个变化程度还不会难解释。

\begin{Shaded}
\begin{Highlighting}[]
\NormalTok{hemoglobin}\SpecialCharTok{$}\NormalTok{血红蛋白数值变化 }\SpecialCharTok{\%\textgreater{}\%} \FunctionTok{summary}\NormalTok{()}
\end{Highlighting}
\end{Shaded}

\begin{verbatim}
##    Min. 1st Qu.  Median    Mean 3rd Qu.    Max.    NA's 
## -90.400  -6.000  13.000   6.225  21.500  75.000      13
\end{verbatim}

\end{document}
